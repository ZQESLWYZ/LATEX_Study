\documentclass{article} % 文档类
\usepackage{ctex} % 中文语言支持的宏包

% 添加标题、作者和日期,除了下面需要在正文区加 \maketitle
% 注意 \and 和 \thanks 的用法

\title{\LaTeX 《西游记》}
\author{WU CHENGEN\thanks{Email:cswyz@stu.hist.edu.cn}}
\date{\today}
% \begin前的是导言区,就是文档类和宏包这些

\begin{document}
% 正文区
\maketitle
 
% \hspace{1em} 是水平空一个汉字宽度的空格
% \vspace{1em} 是垂直空一个汉字宽度的空格

第一回\hspace{0.5em}灵根育孕源流出\hspace{0.5em}心性修持大道生\\
% \\ 或 \newline是换行
诗曰:\vspace{0.5em}\\混沌未分天地乱,茫茫渺渺无人见。\\
自从盘古破鸿濛,开辟从兹清浊辨。\\
覆载群生仰至仁,发明万物皆成善。\\
欲知造化会元功,须看《西游释厄传》。

% 文本间空一行或多行会当初新段落处理,自动首行缩进

盖闻天地之数,有十二万九千六百岁为一元。将一元分为十二会,乃子、丑、寅、卯、辰、巳、午、未、申、酉、戌、亥之十二支也。每会该一万八百岁。且就一日而论:子时得阳气,而丑则鸡鸣;寅不通光,而卯则日出;辰时食后,而巳则挨排;日午天中,而未则西蹉;申时晡,而日落酉,戌黄昏,而人定亥。譬于大数,若到戌会之终,则天地昏曚而万物否矣。再去五千四百岁,交亥会之初,则当黑暗,而两间人物俱无矣,故曰混沌。又五千四百岁,亥会将终,贞下起元,近子之会,而复逐渐开明。邵康节曰:“冬至子之半,天心无改移。一阳初动处,万物未生时。”到此,天始有根。再五千四百岁,正当子会,轻清上腾,有日,有月,有星,有辰。日、月、星、辰,谓之四象。故曰,天开于子。又经五千四百岁,子会将终,近丑之会,而逐渐坚实。《易》曰:“大哉乾元!至哉坤元!万物资生,乃顺承天。”至此,地始凝结。再五千四百岁,正当丑会,重浊下凝,有水,有火,有山,有石,有土。水、火、山、石、土,谓之五形。故曰,地辟于丑。又经五千四百岁,丑会终而寅会之初,发生万物。历曰:“天气下降,地气上升;天地交合,群物皆生。”至此,天清地爽,阴阳交合。再五千四百岁,正当寅会,生人,生兽,生禽,正谓天地人,三才定位。故曰,人生于寅。

感盘古开辟,三皇治世,五帝定伦,世界之间,遂分为四大部洲:曰东胜神洲,曰西牛贺洲,曰南赡部洲,曰北俱芦洲。这部书单表东胜神洲。海外有一国土,名曰傲来国。国近大海,海中有一座名山,唤为花果山。此山乃十洲之祖脉,三岛之来龙,自开清浊而立,鸿濛判后而成。真个好山!有词赋为证。赋曰:

\end{document}